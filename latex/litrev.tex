%\section{Literature Review}
%\label{sec:litrev}
\begin{CJK*}{UTF8}{min}

%The theoretical and methodological framework of this study is built on several branches of the economic literature: The first one of these branches consists of studies examining the impact of fiscal policy in general, and government spending in particular, on macroeconomic variables. [... Add studies + summary here]
%The other branch, this exercise relies on, consists of studies employing policy counterfactuals as a method. These have mostly been used to evaluate the impact of monetary policy measures. [... Add studies + summary here]. 

The method employed in this exercise falls into a stream of literature championed by \citet{csims1980} using VAR models to explore the macroeconomic effects of policies with the help of counterfactual forecasts. These methods are mostly  used to study monetary policy scenarios, using the estimated structural VAR (SVAR) model in one regime or period and applying it to an estimated SVAR model for a different one, to contrast model outcomes. Examples of this approach are \citet{csims1998} exploring the role played by monetary policy in the U.S. postwar economy, \citet{csims2006}  examining U.S. inflation in the 1970s and 1980s, and \citet{primi2005} studying the impact of monetary policy on the dynamics of inflation and unemployment in the same period. After the crisis of 2008, policy counterfactuals constructed by an explicitly stated assumed change in one monetary policy variable were contrasted against a no policy forecast, which is constructed using the actually observed realizations of the variable. This approach can be found in \citet{lenza2010} and \citet{kapetanios2012}, who sought to study the impact of quantitative and qualitative easing in the EU and the UK economy by employing a large Bayesian VAR (BVAR) model. In addition to the BVAR model, \citeauthor{kapetanios2012} also used a change-point SVAR and a time-varying parameter VAR.

However, the use of SVAR-based policy counterfactuals and their reliability have been subject to criticism mostly stemming from the Lucas Critique \citep{lucas1976}. Lucas criticized the use of econometric models to analyze policy interventions that do not take into account that economic agents can change their behavior as a consequence of some policy interventions, translating into constant coefficients of the estimated model. \citet{benati2010} also argued that SVAR counterfactuals are less reliable than counterfactual estimates from DSGE models as they fail to properly capturing the consequence of varying parameters in, for example, the Taylor Rule. A literature review by \citet{ericsson1995}, however, found little empirical support for the Lucas Critique.

The Japanese economy after the 1980's has been subject to extensive research in economics. The Bubble Economy of the 1980's, which was fueled by a combination of rising real estate prices caused by the structure of Japan's urban development program, the de-regularization and internationalization of financial markets (see \citealt{oizumi1994}, and \citealt{allen2000} for a more detailed account) and its subsequent burst due to a change in monetary policy  led to a deflationary spiral and a period of weak economic growth. This period is known as the lost decade and became even more relevant as a subject of study, as Western economies had to face similar problems after the financial crisis in 2008. Possible cures against prolonged downturns in the aftermath of such crises proposed by economists include unconventional monetary policy, for example proposed by Okamoto in the form of "quantitative easing of a new dimension" \footnote{original \emph{ijigen no kin'y\={u} kanwa}, 異次元の金融緩和 } \citep{okamoto2013} or by \citet{hausman2014}, expansionary fiscal policy financed by public borrowing (see \citealt{koo2015} for one such proposal), structural reforms for education and welfare \citep{saxonhouse1998}, or the subsidization of productive and innovative firms through incentives \citep{hayashi20021990s}.

More recently \citet{morita2017} analyzed the impact of fiscal policy shocks by employing a VAR model with robust sign restrictions.In his study, he examines how fiscal policy shocks have different impacts on the economy, depending on whether the shocks were anticipated by economic agents or not. He found that the expectations formed by agents do indeed make a difference in how the a fiscal policy shock affects an economy. His model, which uses quarterly data from 1980 until 2015 (Q2) also suggests including excess stock returns for the construction industry to account for the structure of public development projects in Japan. 

The long period of observation that we can leverage in the Japanese case makes it an interesting subject also for this study. The following section will lay out how the following exercise builds on a variation of his model for the Japanese economy and use it to apply the methodology from \citeauthor{kapetanios2012}.

\end{CJK*}
%move to method
%Most of this discussion, however, is centered around the restrictions and the structure imposed on econometric models. The methodology employed by [quote Lenza(2010)] and [quote Kapetanios et al. (2012)], on the other hand offer the practical advantage that they require \emph{reliability of the produced forecasts} imposing only little structure on the data. In such cases, [quote Duy and Thoma (1998)] showed that differenced VAR models perform similarly well as error-correction models with theoretical and estimated cointegrating restrictions