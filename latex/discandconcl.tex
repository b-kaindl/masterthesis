\section{Discussion of the Results and Conclusion}

The results of the counterfactual scenarios allow two main conclusions. First, they indicate that the effects of government spending increases depend on how these spendings are made. In the median, one-time increases fail to significantly alter the growth paths of the macroeconomic variables we included over a period of two years, regardless of whether spending returns to lower levels instantly or is reduced gradually over the subsequent quarters.

Continually sustained increases in government spending, on the other, seem to have a significant positive impact on GDP and consumption growth if they are implemented as a proportional 'markup' on actual expenditure, and a significant positive effect on inflation if the same increase is spread in constant rates over each quarter. Still, these effects were small, especially when they are held against the direct economic costs and the political feasibility to increase government expenditure so drastically from one quarter to another. 

Nonetheless, it is important to keep in mind that the forecast methodology was rather simple and is easily extensible. It imposed little structure on the data, other than the choice of variables and their ordering -- modeling decisions that are difficult to avoid, but can be tested for -- the exogeneity of government expenditure as a fiscal policy instrument, which can be avoided by, for example, by finding and employing appropriate instrumental, that are indeed exogenous. Finding such instruments can also help determine, to which extent governments are free in their fiscal policy choice and is an exercise which can also be primarily data-driven by employing exogeneity employing tests for exogeneity, such as the one proposed by \citet{ericsson1998}, who examined necessary conditions for exogeneity of variables. Another implicit assumption of this model is the time- and policy-insensitive behavior of economic agents. This is reflected in the constant coefficients of the model, which impose the lack of regime changes over a time frame of 35 years. Without imposing any more structure, endogenous regime changes can also be estimated from the underlying data, as in \citet{kim2008}. Employing approaches similar to this exercises could thus help to motivate deeper research into estimated relationships, using larger datasets, or employing more flexible methods, such as the Large Bayesian VAR model employed by \citeauthor{kapetanios2012}.

Similar approaches could also be taken for policy evaluation and forecasting in scenarios, where good forecasts are more important than identifying economic relationships. Employing methods from machine learning can help to tune VAR models to produce good forecasts in situations, when sufficient data is available, or when bootstrapping or other data augmentation methods are available. The approach itself can thus be of interest for aims other than economic research, regardless of whether it is in the context of fiscal policy or other macroeconomic policy measures.