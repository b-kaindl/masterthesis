\section{Introduction}

Following the global financial crisis of 2008, policymakers and economists alike have been looking for policy measures that can help economies alleviate and overcome economic downturns in the aftermath of such events. Especially in times when conventional policies reach their borders, finding and evaluating unconventional instruments and approaches has become more important.

Especially unconventional monetary policies have received major attention, in particular as major economies started employing measures such as negative central bank interest rates and quantitative easing. The effects of unconventional policy measures have been subject to an array of studies employing vector autoregressive models (VAR) and structural VAR (SVAR) models. An especially interesting approach was taken by \citet{kapetanios2012}, who employed forecasts from different VAR models to construct counterfactual scenarios, which are held against a baseline forecast.

For similar reasons, research also took an interest in examining the impact of fiscal policy on a macroeconomic scale. Notably \citet{morita2017} employed an SVAR model to identify the impact of different fiscal policy shocks on the japanese post-bubble economy, leveraging a case that produced a time series of data ranging over three decades. 

However, most of the studies examining fiscal policy impose structure in their models, as their main focus is the precise identification of relationships. The following study aims to contribute to research on fiscal policy measures and their evaluation by following \citeauthor{kapetanios2012} in employing a VAR model to examine the effects of a government spending increase on the Japanese economy. For this exercise, we use the same data as in \citealt{morita2017}.

The first section of this study provides an overview of the literature on methods employing VAR models for policy evaluation. The second section describes the data, the model selection process and the method employed. In the third part, the results are provided and discussed, and the fourth section provides concluding remarks and potential uses for such an approach in further research and policy evaluation.